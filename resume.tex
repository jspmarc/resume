\documentclass[10pt]{article}

%% Load packages %%
\usepackage{
    enumitem,
    fontspec,
    tabularx,
    titling,
}
\usepackage[
    a4paper,
    footnotesep=0cm,
    footskip=0cm,
    headheight=0cm,
    headsep=0cm,
    margin=1cm,
    marginpar=0cm,
    tmargin=0.5cm,
    % showframe=true,
]{geometry}
\usepackage{multicol}
\setlength{\multicolsep}{0pt plus 0pt minus 0pt} % set multicolsep length
\usepackage[
    colorlinks = true,
    urlcolor = RoyalBlue,
]{hyperref}
\urlstyle{same} % set url style
\usepackage{titlesec}
\usepackage[svgnames]{xcolor}

\pagenumbering{gobble}

% Start: set fonts
\setmainfont[
    ItalicFont=Playfair Display Italic,
    BoldFont=Playfair Display Bold,
    BoldItalicFont=Playfair Display Bold Italic,
]{Playfair Display Regular}
\setsansfont[
    % falls back to main font
    ItalicFont=Open Sans Italic,
    BoldFont=Open Sans Bold,
    BoldItalicFont=Open Sans Italic,
]{Open Sans Regular}
% End: set fonts

% Start: formatting section headers
\titleformat{\section}
    {\bfseries\centering\large\rmfamily\titlerule\vspace{0.2em}}
    {}
    {0em}
    {\vspace{0.2em}}[\titlerule]
% End: formatting section headers

% Start: formatting title
\renewcommand{\maketitle}{
    \rmfamily
    \noindent
    {\Huge \theauthor} ---
    \sffamily
    Software Engineer

    \vspace{0.1em}
    \noindent
    \href{mailto:jspmarcello@live.com}{jspmarcello@live.com} $\bullet$
    \href{https://wa.me/6289651512403}{(62)896-5151-2403} $\bullet$
    \href{https://www.linkedin.com/in/josepmk1}{https://www.linkedin.com/in/josepmk1} $\bullet$
    \href{https://www.jspmarc.dev}{https://www.jspmarc.dev}
}
% End: formatting title

% Change default font
\renewcommand{\familydefault}{\sfdefault}

% Start: new work commands
\newcommand{\workExpVspace}{1em}
\newcommand{\workExp}[6]{
    % #1 is company name
    % #2 is job type (internship, part time, full time, contract)
    % #3 is start month/year
    % #4 is end month/year
    % #5 is role
    % #6 is description
    \noindent \begin{tabularx}{\textwidth}{@{}X c|r}
        \textbf{\textsc{#5}} (#2) & \textbf{#1} & #3---#4
    \end{tabularx}

    {#6}
    \vspace{\workExpVspace}
}
% End: new work commands

% Start: organizational experiences commands
\newcommand{\orgExp}[5]{
    % #1 is role
    % #2 is organization name
    % #3 is start month/year
    % #4 is end month/year
    % #5 is description
    \noindent \textbf{\textsc{#1}}---#2

    {#3}--{#4}

    {#5}
    \vspace{\workExpVspace}
}
% End: organizational experiences commands

% Start: list settings
\setlist[itemize]{nosep}
\renewcommand\labelitemi{$\bullet$}
% End: list settings

% Start: new projects commands
\newcommand{\project}[3]{
    \noindent \href{#2}{#1}

    {#3}
}
% End: new projects commands

\begin{document}

\title{R\'esum\'e}
\author{Josep Marcello}

\maketitle

\section{Introduction}

I'm Josep Marcello, an avid software engineer from Indonesia. I have a keen interest in distributed
and parallel systems which leads to me loving building scalable software and systems. Even though I
my main interest is parallel and distributed systems, I have a large experience in web development,
especially in back-end which is used by both web applications and mobile applications. I also have
a deep attention to detail to keep code clean and be as resource optimized as possible.

\section{Working Experience}

\workExp
    {Tiket.com}
    {Full time}
    {October 2022}
    {now}
    {Software Engineer I}
    {
        \begin{itemize}
            \item Tech stack: Java 8 (and RxJava), Java 11 (and Reactive Java), Maven, Spring,
                MongoDB, Kibana, ElasticSearch.
            \item Work with 2 other software engineers, 1 tech lead to improve and mantain
                all 5 of Tiket.com's accommodation demand post-purchase services.
            \item Interim PIC of multi-currency feature for accommodation demand post-purchase
                team.
            \item Refactor booking API to better prevent double booking.
            \item Help another team by making API contract and implementing the API for their new
                feature using needs data from my team’s service.
            \item Move an outbound API from using gRPC to HTTP.
            \item Add monitoring for kafka delay and gRPC latency metric.
            \item Implement a hard deletion API (using Kafka) to hard delete user data from one of
                my team’s service.
            \item Optimize CPU usage by refactoring reactive Java code.
            \item Experiment on usage of Quarkus Java for better resource optimization and shared
                my result with the accommodation team.
            \item Work on other bug fixes and adjustments along with other software engineers,
                QAs, and product manager.
        \end{itemize}
    }

\workExp
    {Tiket.com}
    {Internship}
    {June 2022}
    {October 2022}
    {Back end Engineer}
    {
        \begin{itemize}
            \item Tech stack: Java 8 (and RxJava), Java 11 (and Reactive Java), Maven, Spring,
                MongoDB, Kibana, ElasticSearch.
            \item Worked with 3 other software engineers, 1 tech lead to improve and mantain
                Tiket.com's hotel demand post-purchase team services.
            \item Created a Python script to automate deleting unused data from ElasticSearch and
                collaborated with DevOps to deploy it.
            \item Initiated error tracing (both on production and staging) and communicate my
                findings to my lead and coworkers.
            \item Adjusted booking code to allow usage of different order service for B2B and B2C
                bookings.
            \item Implemented a monitoring library for inbound and outbound API response time to
                one of my team’s service.
        \end{itemize}
    }

\workExp
    {Kenangan.com}
    {Part time}
    {December 2021}
    {June 2022}
    {Junior Back end Engineer}
    {
        \begin{itemize}
            \item Tech stack: Node.js, Fastify.js, MySQL, Sequelize.js
            \item Worked with 2 other back end engineers and 2 mobile app engineers to create API
                contracts and adjust existing APIs.
            \item Implemented the whole order flow (from user submitting their order until they
                receive their order).
            \item Created a service to download videos from other content bucket to Kenangan's
                main S3 bucket.
            \item Integrated a courier service API with Kenangan’s back end service to
                automatically update user’s order status.
            \item Redesigned, documented, and rewrote the database schema.
            \item Refactored the back-end code to make it more readable for new joiners.
        \end{itemize}
    }

\workExp
    {ITB Distributed System Laboratory}
    {Part time}
    {July 2021}
    {now}
    {Laboratory Assistant}
    {
        \begin{itemize}
            \item Work with 8 other assitants to create tasks and projects for 4 courses revolving
                around distributed system and low-level programming with around 160 students each.
        \end{itemize}
    }

\workExp
    {CoLearn}
    {Internship}
    {September 2020}
    {February 2021}
    {Data Analyst}
    {
        \begin{itemize}
            \item Optimized machine learning algorithm by evaluating over 10,000 pictures from
                machine learning prediction.
            \item Helped optimize my team's workflow by creating a Python script to automatically
                download JSON data from an API and parse it into a CSV file.
        \end{itemize}
    }

\workExp
    {Wardaya College}
    {Internship}
    {July 2020}
    {August 2020}
    {Software Engineer}
    {
        \begin{itemize}
            \item Tech stack: Python Flask, MySQL
            \item Developed a new e-learning platform with 3 other software engineers.
            \item Implemented the login flow with password salting and hashing.
            \item Designed the database schema for the back-end.
        \end{itemize}
    }

\section{Skills}

\begin{tabularx}{\textwidth}{@{}X l}
\textbf{Programming languages} & Rust, Java, Go, Python, C++, TypeScript, JavaScript, C, C\# \\
\textbf{Frameworks}            & Express.js, React.js, Tailwindcss, Svelte.js, Vue.js, Spring\
                                 Boot, Actix-web \\
\textbf{Development tools}     & SQL Databases, Git, Docker, Redis, InfluxDB, MongoDB, GitHub \&\
                                 GitLab CI/CD \\
\textbf{Languages}             & Indonesian (native), English (fluent)
\end{tabularx}

\section{Projects}

\project
    {Smart Room Conditioner}
    {https://github.com/jspmarc/IF4051-Tubes}
    {
        An IoT app to autoamatically open/close dorm window/door based on room temperature,
        humidity, and CO2 PPM.

        Worked with 2 other software engineers to create a software that can:
        \begin{multicols}{2}
            \begin{itemize}
                \item IoT sensors emits data to MQTT every 1 second.
                \item IoT data is captured by PySpark to calculate statistics (mean, min, and max)
                    then the statistics is published to Kafka.
                \item IoT data is also captured by Telegraf to be saved to InfluxDB.
                \item Statistics published to Kafka is captured by FastAPI service and then given
                    to ML model to determine whether to open door/window or not.
                \item User can override ML prediction.
                \item User can be alerted every 30 minutes using email if door/window state is not
                    the same as ML prediction.
                \item Both the statistics and the sensor data is shown in the user app.
            \end{itemize}
        \end{multicols}

        Tech stack: Technologies used for this project are Python, PySpark,
        FastAPI, aiokafka, Vue.js + Vite.js, Tailwind.css, Chart.js,
        Platform.io (for IoT programming), InfluxDB, Telegraf, Redis, MQTT,
        Kafka, Docker
    }

\project
    {Simple Message Queue}
    {https://github.com/jspmarc/simple-message-queue}
    {
        A specification and implementation of simple message queue.

        In the implementation, multiple clients can connect to a queue server to push and pull
        messages.

        Tech stack: Rust
    }

\project
    {Pendekin-V2}
    {https://github.com/hmif-itb/pendekin-v2}
    {
        A link shortener for my student’s association (HMIF)

        This is a rewriting from Pendekin V1 which uses Google to authenticate users.

        Tech stack: Go, Go Fiber, MongoDB, Docker
    }

\project
    {More projects}
    {https://github.com/jspmarc}

\section{Education}

\textbf{Bandung Institute of Technology}---Bandung, Indonesia

\textbf{Bachelor of Computer} (Informatics Engineering)

\textbf{GPA}: 3.56/4.00

\textbf{Notable finished Courses}:
\begin{multicols}{2}
    \begin{itemize}
        \item Web-based Application Development
        \item Software Project
        \item Parallel and Distributed System
        \item Database Management
        \item Computer Networks
        \item Distributed Application Development
    \end{itemize}
\end{multicols}

\textbf{Final project}: Hardware-in-the-loops (HILS) Connector for Autonomous Vehicle Simulation

\section{Organizational Experiences}

\orgExp
    {Head of Technology, Issue, and Exploration (TIE) Department}
    {HMIF ITB}
    {May 2022}
    {May 2023}
    {
        \begin{itemize}
            \item Lead and managed 3 divisions under TIE Department.
            \item Worked with other divisions to provide internal HMIF services and software (Line
                Bot, Discord Bot, and link shortener).

                Tech used: Go, JavaScript/TypeScript, Svelte.js, Python, Docker, Nginx, Google
                Workspace Suite, GitHub project
        \end{itemize}
    }
    
\orgExp
{Head of Web Development Division}
{Parade Wisuda Juli 2021 ITB}
{May 2021}
{July 2021}
{
    \begin{itemize}
        \item Lead 12 software engineers to create the website's design, front-end, and back-end.
        \item Setup CI/CD pipeline and deployment server.
        \item Tech stack: TypeScript, Express.js, Sequelize.js, React.js, SCSS, MariaDB, GitHub
            CI/CD, Docker
    \end{itemize}
}

\section{Achievements}

\begin{itemize}
    \item Finalist at Bina Nusantara Cyber Security Community (Binus CSC)
        National Cyber Week, Cyber Arena CTF Competition.
    \item Finalist at UGM JOINTS 2021 CTF Competition.
    \item Participant at TechnoScape 2021 Hackathon Competition.
\end{itemize}

\end{document}
